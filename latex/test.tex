%pdflatex  test.tex




\documentclass[UTF8, 12pt, a4paper, oneside, titlepage]{book}

 %% \documentclass[UTF8, a4paper, oneside, titlepage]{book}
\usepackage{amsmath}
\usepackage{ctex}
\usepackage{url}
\usepackage{color}
%\setcounter{secnumdepth}{0}
%\raggedright

%% \renewcommand{\abstractname}{Summary for test}

\begin{document}


\title{\LaTeX Test}
\author{Jack\\Rose}
\date{\color{blue}2013-02-17}

  \maketitle
  
  %% \begin{abstract}
  %% This is abstract.
  %% \end{abstract}
  \tableofcontents
  
  \LaTeX{} is a document preparation system for the \TeX{}
  typesetting program. It offers programmable desktop publishing
  features and extensive facilities for automating most aspects of
  typesetting and desktop publishing, including numbering and
  cross-referencing, tables and figures, page layout, bibliographies,
  and much more. \LaTeX{} was originally written in 1984 by Leslie
  Lamport and has become the dominant method for using \TeX; few
  people write in plain \TeX{} anymore. The current version is
  \LaTeXe.
 
  % This is a comment; it will not be shown in the final output.
  % The following shows a little of the typesetting power of LaTeX:
  \begin{align}
    E &= mc^2                              \\
    m &= \frac{m_0}{\sqrt{1-\frac{v^2}{c^2}}}
  \end{align}

	\section{First Document}
	This is a exmple.
	中文
	\begin{itemize}
	\item ABC
	\item CBA
	\item 123
	\item 321
	\end{itemize}
	
	\subsection{More information}
	
	
	\begin{center}
		\fbox{\textit{Have a nice day!}}
		
	\end{center}
	
	
	
	\textit{Great?}
	\underline{Underline}
	\begin{flushleft}
	Flush Left
	\end{flushleft}
	
	\begin{center}
	Stand Center
	\end{center}
	
	\begin{flushright}
	Flush Right
	\end{flushright}
	
%	\begin{picture}(20,20)
%	content...
%		\circle
%	\end{picture}

%\chapter{Ch1:第一章}
第一章内容
\section[Effect on staff turnover]{An analysis of the 
effect of the revised recruitment policies on staff 
turnover at divisional headquarters}
第二章

\begin{itemize}

\item Itemized lists usually have a bullet;

\item Long items use `hanging indentation', 
whereby the text is wrapped with a margin 
which brings it clear of the bullet used in 
the first line of each item;

\item The bullet can be changed for any other 
symbol, for example from the \textsf{bbding} 
or \textsf{pifont} package.

\end{itemize}

\url{http://www.ctan.org/tex-archive/info/beginlatex/}

\end{document}

This is also comment.
